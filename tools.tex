%%%%%%%%%%%%%%%%%%%%%%%%%%%%%%%%%%%%%%%%%%%%%%%%%%%%%%%%%%%%%%%%%%%%%%%%%%%%
%                                                                          %
% Copyright (c) 2018 eBay Inc.                                             %
%                                                                          %
% Licensed under the Apache License, Version 2.0 (the "License");          %
% you may not use this file except in compliance with the License.         %
% You may obtain a copy of the License at                                  %
%                                                                          %
%  http://www.apache.org/licenses/LICENSE-2.0                              %
%                                                                          %
% Unless required by applicable law or agreed to in writing, software      %
% distributed under the License is distributed on an "AS IS" BASIS,        %
% WITHOUT WARRANTIES OR CONDITIONS OF ANY KIND, either express or implied. %
% See the License for the specific language governing permissions and      %
% limitations under the License.                                           %
%                                                                          %
%%%%%%%%%%%%%%%%%%%%%%%%%%%%%%%%%%%%%%%%%%%%%%%%%%%%%%%%%%%%%%%%%%%%%%%%%%%%

\section{\texttt{dsinfo} -- Dataset Information}
The \texttt{dsinfo} command line tool gives a compact, but easy to
read, overview of a dataset or a dataset chain.  The tool is located
in the \texttt{accelerator} home directory, and the full file name is
\texttt{dsinfo.py}.  Example invocation
\begin{shell}
% ./dsinfo.py test-20
\end{shell}
The argument can be one or more jobids or dataset ids.  If the
argument is a jobid, it is assumed that the dataset name is
\texttt{default}.  If there are more than one dataset in the job, a
list of dataset names will be returned.


\section{\texttt{dsgrep} -- Find Data in Dataset}
The \texttt{dsgrep} command line tool is used to look at datasets or
dataset chains.
\subsection{Invocation}
\begin{shell}
dsgrep.py [options] pattern ds [ds [...]] [column [column [...]]
\end{shell}
The \texttt{pattern} is a regular expression and \texttt{ds} are datasets.  For example
\begin{shell}
dsgrep.py Alice test-0 test-1/special name
\end{shell}
Will look for the string \texttt{Alice} in the \texttt{name} column of
the two datasets \texttt{text-0} and \texttt{test-1/special}.
Optional arguments are
\begin{snugshade}
  \begin{tabular}{p{4cm}p{9cm}}
      \texttt{-h}\hspace{3cm}\texttt{--help} & show help message and exit\\[4ex]
      \texttt{-c}\hspace{3cm}\texttt{--chain} & follow dataset chains\\[4ex]
      \texttt{-i}\hspace{3cm}\texttt{--ignore-case} & Case insensitive pattern\\[4ex]
  \end{tabular}
\end{snugshade}
Strings and columns with special characters have to be quoted.

\subsection{Abuse dsgrep to show datasets}
The data in a dataset may be printed to \texttt{stdout} by
\texttt{grep}ing using a regexp that always matches, like this
\begin{shell}
./dsgrep.py . test-0 | less
\end{shell}
