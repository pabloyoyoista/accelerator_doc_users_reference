%%%%%%%%%%%%%%%%%%%%%%%%%%%%%%%%%%%%%%%%%%%%%%%%%%%%%%%%%%%%%%%%%%%%%%%%%%%%
%                                                                          %
% Copyright (c) 2018 eBay Inc.                                             %
%                                                                          %
% Licensed under the Apache License, Version 2.0 (the "License");          %
% you may not use this file except in compliance with the License.         %
% You may obtain a copy of the License at                                  %
%                                                                          %
%  http://www.apache.org/licenses/LICENSE-2.0                              %
%                                                                          %
% Unless required by applicable law or agreed to in writing, software      %
% distributed under the License is distributed on an "AS IS" BASIS,        %
% WITHOUT WARRANTIES OR CONDITIONS OF ANY KIND, either express or implied. %
% See the License for the specific language governing permissions and      %
% limitations under the License.                                           %
%                                                                          %
%%%%%%%%%%%%%%%%%%%%%%%%%%%%%%%%%%%%%%%%%%%%%%%%%%%%%%%%%%%%%%%%%%%%%%%%%%%%


\section{\texttt{daemon} Accelerator Server}

The \texttt{daemon.py} is the main Accelerator daemon/server.  This
program will run in the background and serve \texttt{runner} requests...

\subsection{Invocation}

\begin{verbatim}
daemon.py [-h] [--debug] [--config CONFIG_FILE] [--port PORT | --socket SOCKET]
\end{verbatim}

Optional arguments
\begin{snugshade}
\begin{tabular}{p{4cm}p{9cm}}
  \texttt{-h}\hspace{3cm}\texttt{---help} & show help message and
  exit.\\[4ex]

  \texttt{---debug} & Start in debug mode.  See section~\ref{sec:debugflag}\\[2ex]
  \texttt{---config CONFIG\_FILE} & configuration file, default
  \texttt{../conf/framework.conf}\\[4ex]

  \texttt{---port PORT} & listen on TCP port (default \pyNone)\\[4ex]

  \texttt{---socket SOCKET} & listen on unix socket, default
  \texttt{socket.dir/default}\\[4ex]
\end{tabular}
\end{snugshade}
The Accelerator and Runner will connect using a unix socket by
default.  There is no need to configure anything.  Setting a port will
make communication happen over that port instead.




\vspace{2cm}
\section{\texttt{urd}  Job Database Server}
Urd is currently run as a stand-alone server.
\subsection{Invocation}
\begin{shell}
urd.py [-h] [--port PORT] [--path PATH]
\end{shell}

\begin{snugshade}
\begin{tabular}{p{4cm}p{9cm}}
  \texttt{-h}\hspace{3cm}\texttt{---help} & show help message and
  exit.\\[4ex]

  \texttt{---port PORT} & listen on TCP port\\[4ex]

  \texttt{---path PATH} & path to database\\[4ex]
\end{tabular}
\end{snugshade}



\clearpage
\section{\texttt{runner} Build Script Runner}
\label{sec:exec_runner}

The \texttt{runner} is used to execute build scripts.

\subsection{Invocation}
Runner is invoked like this
\begin{shell}
automatarunner.py [options] [script]
\end{shell}
assuming the current work directory is the \texttt{Accelerator}
directory.  The \texttt{script} is either a filename, or the suffix to
a filename starting with \texttt{automata\_}.


When the \texttt{runner} starts, it will first instruct the
Accelerator to scan all method directories to see if there are any new
or changed methods.  Thereafter, the Accelerator will proceed and scan
all source workdirs to see if any new jobs have been created (by
another Accelerator daemon).  Thereafter, it will execute the build
script.

\begin{snugshade}
\begin{tabular}{p{4cm}p{9cm}}
  \texttt{-h}\hspace{3cm}\texttt{---help} & show help message and
  exit.\\[4ex]

  \texttt{-p PORT }\hspace{3cm}\texttt{---port=PORT} & Accelerator
  listening port\\[4ex]

  \texttt{-H HOSTNAME}\hspace{3cm}\texttt{---hostname=HOSTNAME} &
  framework hostname\\[4ex]
  
  \texttt{-S SOCKET}\hspace{3cm}\texttt{---socket=SOCKET} &
  Accelerator unix socket (default
  \texttt{./socket.dir/default})\\[4ex]

  \texttt{-s SCRIPT}\hspace{1cm}\texttt{---script=SCRIPT} & build
  script to run. \texttt{package/automata\_<SCRIPT>.py}.  Defaults to
  ``\texttt{automata}''.  Can be bare arg too.\\[2ex]

  \texttt{-A}\hspace{3cm}\texttt{---abort} & abort (fail) current
  job(s).\\[4ex]

  \texttt{-P PACKAGE}\hspace{3cm}\texttt{---package=PACKAGE} & Run
  build script from this method directory.  Useful if the same script
  name exists in several method directories, for example for testing
  purposes.\\[2ex]

  \texttt{-f FLAGS}\hspace{3cm}\texttt{---flags=FLAGS} & Comma
  separated list of flags, exposed as the set \texttt{urd.flags} in
  build script.\\[2ex]
  
  \texttt{-q}\hspace{3cm}\texttt{---quick} & skip method updates and
  workdirs checking for new jobs.\\[4ex]

  \texttt{-w}\hspace{3cm}\texttt{---just\_wait} & just wait for running
  job, do not run a build script.\\[4ex]

  \texttt{---verbose=VERBOSE} & verbosity level, one of \texttt{no},
  \texttt{status}. \texttt{dots}, or \texttt{log}.\\[2ex]

  \texttt{---quiet} & same as \texttt{---verbose=no}\\[2ex]

  \texttt{---horizon=HORIZON} & Time horizon - dates after this are
  not visible in \texttt{urd.latest}.\\[4ex]
\end{tabular}
\end{snugshade}
To run a build script \texttt{automata\_myscript}, do
\begin{python}
./automatarunner myscript
\end{python}
This works as long as the name of the build script is unique, that is,
it exists in only one method directory.  If not, the method directory
can be specified using the \texttt{-P} option.

A build script named \texttt{automata.py} in a method
directory \texttt{dev} can be launched by
\begin{shell}
% ./automatarunner -P dev
\end{shell}



\subsection{Authorization to Urd}
Authorisation to Urd could be set in the \texttt{URD\_AUTH}
environment variable.  A common way to invoke the runner with Urd
authorisation is like this
\begin{shell}
% URD_AUTH=user:passwd ./runner [script]
\end{shell}
Note that the purpose of the authentication is
actually \textsl{identification}.  It is used to get write access to
certain Urd lists.  Nothing more.









%\clearpage
\section{\texttt{dsinfo} -- Dataset Information}
The \texttt{dsinfo} command line tool gives a compact, but easy to
read, overview of either
\begin{itemize}
\item[] a dataset,
\item[] a chain of datasets, or
\item[] available datasets in a job directory.
\end{itemize}
The tool is located in the \texttt{accelerator} home directory, and
the full file name is \texttt{dsinfo.py}.

\subsection{Invocation}
\begin{shell}
dsinfo.py [-h] [-c] [-l] [-q] [-s] [-S] dataset [dataset ...]
\end{shell}
The \texttt{dataset} option is either a \textsl{dataset}, when used
with the \texttt{-s}, \texttt{-S}, and \texttt{-c} options, or a
\textsl{jobid} when used with \texttt{-l} option.  Datasets or jobids
could be either names or absolute paths.  Examples of valid datasets
are \texttt{test-2}, \texttt{test-2/default}, and
\texttt{/home/wdirs/test/test2/dsx}.  Of these, only \texttt{test-2}
is a valid jobid.  Here are all options
\begin{snugshade}
\begin{tabular}{p{4cm}p{9cm}}
  \texttt{-h}\hspace{3cm}\texttt{---help} & show help message and exit.\\[4ex]
  \texttt{-q}\hspace{3cm}\texttt{---quiiet} & Silently ignore any error.\\
\end{tabular}
\end{snugshade}
When \texttt{dsinfo} is fed with dataset(s)
\begin{snugshade}
\begin{tabular}{p{4cm}p{9cm}}
  \texttt{-c}\hspace{3cm}\texttt{---chain} & Print name and number of
  lines for all datasets in the chain.\\[4ex]
  \texttt{-s}\hspace{3cm}\texttt{---slices} & Print absolute and
  relative number of lines per slice for the input dataset.\\[4ex]
  \texttt{-S}\hspace{3cm}\texttt{---chain} & Same as \texttt{-s}, but
  data is for the whole chain of datasets.\\
\end{tabular}
\end{snugshade}
When \texttt{dsinfo} is fed with jobid(s)
\begin{snugshade}
\begin{tabular}{p{4cm}p{9cm}}
  \texttt{-l}\hspace{3cm}\texttt{---list} & Print the name and number
  of lines of all datasets in the input jobid.\\
\end{tabular}
\end{snugshade}

\begin{verbatim}
WIDE37-1/default
    Parent: None
    Hashlabel: None
    Columns:
        'a string'       ascii
        'gauss float'    float64
        'gauss number'   number
        'large number'   number
        'small integer'  int32
        'small number'   number
    6 columns
    1,000,000 lines
    Chain length 2, from WIDE37-0 to WIDE37-1
    0: WIDE37-0/default (1,000,000)
    1: WIDE37-1/default (1,000,000)
    Balance, lines per slice, full chain:
          0:   2.70% (54,056)   13:   2.70% (54,054)   26:   2.70% (54,054)
          1:   2.70% (54,054)   14:   2.70% (54,054)   27:   2.70% (54,054)
          2:   2.70% (54,054)   15:   2.70% (54,054)   28:   2.70% (54,054)
          3:   2.70% (54,054)   16:   2.70% (54,054)   29:   2.70% (54,054)
          4:   2.70% (54,054)   17:   2.70% (54,054)   30:   2.70% (54,054)
          5:   2.70% (54,054)   18:   2.70% (54,054)   31:   2.70% (54,054)
          6:   2.70% (54,054)   19:   2.70% (54,054)   32:   2.70% (54,054)
          7:   2.70% (54,054)   20:   2.70% (54,054)   33:   2.70% (54,054)
          8:   2.70% (54,054)   21:   2.70% (54,054)   34:   2.70% (54,054)
          9:   2.70% (54,054)   22:   2.70% (54,054)   35:   2.70% (54,054)
         10:   2.70% (54,054)   23:   2.70% (54,054)   36:   2.70% (54,054)
         11:   2.70% (54,054)   24:   2.70% (54,054)
         12:   2.70% (54,054)   25:   2.70% (54,054)
    Max to average ratio: 1.000
    2,000,000 total lines in chain
\end{verbatim}


Example invocation
\begin{shell}
% ./dsinfo.py test-20
\end{shell}
The argument can be one or more jobids or dataset ids.  If the
argument is a jobid, it is assumed that the dataset name is
\texttt{default}.  If there are more than one dataset in the job, a
list of dataset names will be returned.


Note also that it may be convenient to use the shells feature....
\begin{shell}
% dsinfo -l -q test-{0..99}
\end{shell}
to scan for datasets in the 100 first jobids.  Adding the \texttt{-q}
option will make \texttt{dsinfo} suppress the warning messages for
those jobids that do not contain datasets.




\section{\texttt{dsgrep} -- Find Data in Dataset}
The \texttt{dsgrep} command line tool is used to look at datasets or
dataset chains.
\subsection{Invocation}
\begin{shell}
dsgrep.py [options] pattern ds [ds [...]] [column [column [...]]
\end{shell}
The \texttt{pattern} is a regular expression and \texttt{ds} are datasets.  For example
\begin{shell}
% dsgrep.py Alice test-0 test-1/special name
\end{shell}
Will look for the string \texttt{Alice} in the \texttt{name} column of
the two datasets \texttt{text-0} and \texttt{test-1/special}.
Optional arguments are
\begin{snugshade}
  \begin{tabular}{p{4cm}p{9cm}}
      \texttt{-h}\hspace{3cm}\texttt{--help} & show help message and exit\\[4ex]
      \texttt{-c}\hspace{3cm}\texttt{--chain} & follow dataset chains\\[4ex]
      \texttt{-i}\hspace{3cm}\texttt{--ignore-case} & Case insensitive pattern\\[4ex]
  \end{tabular}
\end{snugshade}
Strings and columns with special characters have to be quoted.



\subsection{Abuse dsgrep to show datasets}
The data in a dataset may be printed to \texttt{stdout} by
\texttt{grep}ing using a regexp that always matches, like this
\begin{shell}
% ./dsgrep.py . test-0 | less
\end{shell}
