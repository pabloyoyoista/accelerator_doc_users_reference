%\section{Other Useful Functions}
%The \texttt{extras} module contains a set of useful functions

\section{Share Data Between Jobs:  the \texttt{blob} Module}

The simplest way to share reasonable amounts of data between jobs is
by using the \texttt{blob} module.  Note that the Accelerator will set
the ``current work directory'' to the current job directory when
building a method, so all files created by a job will be stored in the
current job directory, unless the filename contains a path pointing
elsewhere.

\subsection*{Storing a  Single File}
Data is saved in this way
\begin{python}
import blob
def synthesis():
    data = ...  # some data created here
    blob.save(data, filename)
\end{python}
The data is loaded like this
\begin{python}
import blob
def synthesis()
    blob.load(filename)
\end{python}



\subsection{Storing a Sliced File}
It is also possible to use the \texttt{blob} module in \analysis.
From a user's perspective it will look like a single file is being
handled, but there is actually one file per slice.  This is how to do
it
\begin{python}
def analysis(sliceno):
    # save data in slices like this
    blob.save(data, filename, sliceno=sliceno)
    # load like this
    data = blob.load(filename, sliceno=sliceno)
\end{python}



\subsection{Save Files for Debugging}
There is one more argument, \texttt{temp}, which controls persistence
of the files.  It is by default set to \mintinline{python}/False/,
which implies that the stored file is not temporary.  But setting it
to \mintinline{python}/True/, like in the following
\begin{python}
    blob.save(data, filename, temp=True)
\end{python}
will cause the stored file to be deleted upon job completion.  The
argument takes two additional values, \texttt{DEBUG} and
\texttt{DEBUGTEMP}, working like this
\begin{itemize}
\item[] \texttt{DEBUG} -- file will be stored \emph{only} in debug
  mode
\item[] \texttt{DEBUGTEMP} -- file will always be stored, but
  \emph{removed} upon job completion only in debug mode.
\end{itemize}
Example
\begin{python}
from extras import Temp
def analysis(sliceno):
  # save only if --debug
  blob.save(data, filename, sliceno=sliceno, temp=Temp.DEBUG)
  # save always, but remove unless --debug
  blob.save(data, filename, sliceno=sliceno, temp=Temp.DEBUGTEMP)
\end{python}



\section{Find the Full Path of a File in Another Job}
Accessing a file stored in another job from within a method or build
script is simple, and the functionality is implemented
in \texttt{resolve\_jobid\_filename()}.  The function takes two
arguments, a \textsl{jobid} and a \textsl{filename}.  See the example
below
\begin{python}
from jobid import resolve_jobid_filename

jobids = ('oldjob',)

def synthesis():
    filename = resolve_jobid_filename(jobids.oldjob, 'nameoffile')
\end{python}
Note that this function works in a build script as well.



\section{Symlinking}
Creating a symlink, for example from the \texttt{result\_directory} to
current workdir, may be implemented in a very simple and safe way like
this
\begin{python}
from extras import symlink

def synthesis(RESULT_DIRECTORY):
    ...
    with open(filename, 'wb') as fh:
         fh.write(...)
    symlink(filename, RESULT_DIRECTORY)
\end{python}
The \texttt{extras.symlink} function will write a soft link
to \texttt{filename} in \texttt{RESULT\_DIRECTORY}, overwriting it if
it already exists.

\subsection{job\_params}

\subsection{job\_post}

\subsection{json\_encode}

\subsection{json\_decode}

\subsection{json\_save}

\subsection{json\_load}



\subsection{blob\_load}
tar option \texttt{default=} som returneras ifall filen inte finns,
se koden.  Verkar som att det bombar om filen inte finns och default är None.
(Så defaultvärdet kan inte vara None alltså.)


\subsection{blob\_save}

\subsection{DotDict}

\subsection{OptionEnum}

\subsection{OptionString}

\subsection{RequiredOption}

\subsection{OptionDefault}

\subsection{gzutil}
\begin{python}
with gzutil.GzUnicodeLines(filename, strip_bom=True) as fh:
\end{python}
