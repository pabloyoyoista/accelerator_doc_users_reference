\section{Dataset Iterators}

Iterators are used to stream dataset data one line at a time into a
method.  Data may be streamed from a single dataset, or from a chain
of datasets, one dataset at a time.


\subsection{Basic Iteration}


\subsubsection*{A Simple Parallel Iterator Invocation}
Read variables X and Y in all slices in parallel

\begin{python}
def analysis(sliceno):
    h = defaultdict(set)
    for user, item in iterate_chain(sliceno, columns=('user', 'item',)):
        h[user].add(item)
\end{python}
data is iterated over all rows and all datasets in the chain, running
in increasing jobid direction.


\subsubsection*{Iterate over all data in synthesis}
It is possible to iterate over all slices in a single loop by
specifying sliceno=None as follows

\begin{python}
def synthesis():
    for user, item in iterate_chain(None, columns=('user', 'item',)):
        h[user].add(item)
\end{python}
slices will be iterated one at at time in increasing order.


\subsection{Halting Iteration}

Sometimes, one wish to limit the number of iterations.  There are
several mechanisms for that:
\begin{itemize}
\item stop\_jobid = jobid
\item stop\_jobid = {jid: datasetname}
  \# ex:\{jid: "source"\} stoppar på job\_params(jid).datasets.source
\item length=N
\item range={column: (min, max)}  \# sloppy\_range
\end{itemize}

The first will iterate until a particular jobid is found.  The second
will iterate until reaching a jobid found as input to another job.
The third will iterate a fixed number of datasets.  The fourth will
iterate a specified range of values in a sorted column.  The
sloppy\_range means that all affected datasets will be iterated, which
is more data but without a lot of filtering overhead and thus faster.

Iteration will be over the shortest range of the conditions above.

It is possible to iterate the dataset chain in the backwards direction
by specifying reverse=True, but note that iteration is always in the
forward direction within each dataset.

% def iterate_chain(self, sliceno, columns=None, length=-1, reverse=False, hashlabel=None, stop_jobid=None, pre_callback=None, post_callback=None, filters=None, translators=None):



\subsection{iterate\_list}
PLACEHOLDER
%\begin{python}
%  # def iterate_list(sliceno, columns, jobids, hashlabel=None, pre_callback=None, post_callback=None, filters=None, translators=None):
%  from dataset import Dataset
%  for item in Dataset.iterate_list(None, None, datasets.source):
%    print item
%\end{python}


\subsection{Translators}

Translators transform data values. A translator is either a callable or
a \texttt{dict}.  Translators are similar to \texttt{filters}, and
always executed before filtering.

The idea behind translators and filters is that they provide a way to
modify code behaviour by supplying functions as iterator options.  In
this way, it is possible to write re-useable simple methods that still
could be altered significantly in behaviour.



\subsubsection*{Callable Translator}

A translator function is called with a tuple and is expected to return
a tuple of the same length.  Using translator functions it is possible
to mix different columns with eachother before sending them to the
iterator output.  Here is an example

\begin{python}
def merger(user, item):
    return "%s:%s" % (user, item), None
for merge, _ in iterate_datasetchain(None, ('user', 'item',), tip_jobid=jobid,
                                     translator=merger):
    ...
\end{python}
\noindent The purpose of this translator is to convert each
\texttt{(user, item)} tuple to a string \texttt{user:item}.  This is
the first output of the translator and iterator and is stored in the
\texttt{merge} variable.  The second output variable is not used in
this application, but a variable still has to be assigned.



\subsubsection*{Translator dict}

One or more columns may be translated independently using a translator
dictionary, specified as \texttt{\{name:\ translation\}}.  A
translation may be either a \texttt{dict} or a callable.  Examples of
both kinds are shown below.

First an example illustrating the use of a translation \texttt{dict}.
Here, integers are translated into more comprehensible strings.

\begin{python}
mapper = {2: 'HUMANLIKE', 4: 'LABRADOR', 5: 'SEASTAR',}
for animal in iterate_datasetchain(None, 'NUM_LEGS', tip_jobid=jobid,
                                     translator={'NUM_LEGS': mapper,}):
    ...
\end{python}
\noindent Items missing in a translation-\texttt{dict} yield
\texttt{None}.

Second, an example of a translation callable.  In this example, each
user string is output from the iterator reads backwards.

\begin{python}
def reverse(x):
    return x[::-1]
for resu, item in iterate_datasetchain(None, ('USER', 'ITEM',), tip_jobid=jobid,
                                       translator={'USER': reverse,}):
    ...
\end{python}



%\newpage
\subsection{Filters}

Filters are run \emph{after} translators.

Filters decide which rows to pass to the iterator output.  A filter is
either a callable or a \texttt{dict}.


\subsubsection*{Callable Filter}

Callable filters receive the iterator tuple as input.  The output must
be \texttt{True} for the tuple to be output from the iterator,
otherwise the iterator continues reading the next line.

The following two examples, of which the first uses a callable filter,
are functionally equivalent

\begin{python}
# Ex1.  filter users that are bearded and longhaired
filter=lambda line: line[1] and line[2]
for user, beard, hair in iterate_datasetchain(None, ('user', 'bearded', 'longhaired'),
                                       tip_jobid=jobid, filters=filter):
    ...

# Ex2.  filter users that are bearded and longhaired
for user, beard, hair in iterate_datasetchain(None, ('user', 'bearded', 'longhaired'),
                                       tip_jobid=jobid):
    if beard and hair:
        ...
\end{python}




\subsubsection*{Filter Dict}

It is possible to filter on one or more columns independently using a
\texttt{dict}.  All filters must be \texttt{True} for a line to be
output from the iterator.  Here are two examples.  The first example
will remove all lines except the ones with valid users.  The second
example will only keep lines with movie items.  (The fact that book is
false is reduntant --- missing keys will result in discarded lines.)

\begin{python}
# keep valid users only
validusers={'user1', 'user2', 'user3'}
filters={'user': validusers.__contains__}

# keep valid users with movie items
validitems={'movie': True, 'book': False}
filters={'user': validusers.__contains__, 'item': validitems.get}
\end{python}




\subsubsection*{Filter by Column Values}


Column values could also be used directly, i.e. the values get
evaluated by Python.  For example, assume there is a column
\texttt{hasbeard} with values being boolean integers, i.e.  \texttt{1}
or \texttt{0}.  Bearded users may be collected by
\begin{python}
for user, beard in iterate_datasetchain(None, ('user', 'bearded',), tip_jobid=jobid,
                                       filters={'bearded': None}):
    bearded.add(user)
\end{python}
\noindent This may seem strange at first, but it works because the key
for the \texttt{bearded} column exists, and the value is
\texttt{None}, and not a callable nor a \texttt{dict}.




%You can also pass a single name (a str) as names_list, in which case
%you don't get a tuple back (just the
%values). Tuple-filters/translators also get just the value in this
%case (column versions are unaffected).

    


\subsection{Callback}
\label{sec:callback}

It is also possible to inject callback functions, either before or
after each dataset is loaded.  If \texttt{sliceno=None},
i.e. iteration runs over all slices of all datasets, it is even
possible to have callback between slice change.

The example below will print dataset \texttt{jobid} for each dataset
prior to iterating over it.

\begin{python}
# pre_callback once per dataset
def prefun(jobid):
    print jobid
for user, item in iterate_datasetchain(sliceno, ('user', 'item',), tip_jobid=jobid,
                                       pre_callback=prefun):
    ...
\end{python}

\noindent Next is an example of an iterator running over all slices.
The callback function is executed before each new slice is iterated.
Note the difference between this example and the previous.  The
callback function in this example takes two arguments, while the
previous takes only one.

\begin{python}
# callback once per slice
def prefun(jobid, sliceno):
    print sliceno, jobid
for user, item in iterate_datasetchain(None, ('user', 'item',), tip_jobid=jobid,
                                       pre_callback=prefun):
    ...
\end{python}
\noindent Post-callback is defined similarly.



\subsubsection*{Skipping Datasets and Slices from Callbacks}
It is possible to skip iterations by raising exceptions, as follows

\begin{python}

# Raise this in pre_callback to skip iterating the coming slice
# (if your callback doesn't want sliceno, this is the same as SkipJob)
raise SkipSlice

# Raise this in pre_callback to skip iterating the coing job
raise SkipJob

# Raise this to quit iterating, but with the side effect that
# post_callback will not run.
# @@@ TO BE DEFINED
raise StopIteration
\end{python}
