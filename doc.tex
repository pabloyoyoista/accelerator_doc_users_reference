\documentclass[a4paper]{report}
\usepackage{graphicx}
\usepackage{color}
%% Replace serif font with (postscript) helvetica
%\usepackage[scaled]{helvet}
\renewcommand*\familydefault{\sfdefault} %% Only if the base font of the document is to be sans serif
\usepackage[T1]{fontenc}
\setlength{\textwidth}{160mm}
\setlength{\oddsidemargin}{0mm}
\setlength{\evensidemargin}{0mm}
\setlength{\textheight}{250mm}
\setlength{\voffset}{-20mm}
%\usepackage{showframe}
\usepackage{xspace}
%\usepackage{pgffor}

%%%  Various reserved words  %%%
%\newcommand{\urd}{\texttt{urd}\xspace}
\newcommand{\joblist}{\texttt{JobList}\xspace}
\newcommand{\jobtuple}{\texttt{JobTuple}\xspace}
%\newcommand{\callmethod}{\texttt{call\_method}\xspace}
%\newcommand{\arecord}{\texttt{a.record}\xspace}
%\newcommand{\automatacommon}{\texttt{automata\_common}\xspace}
%\newcommand{\defaultdict}{\texttt{defaultdict}\xspace}
%\newcommand{\jobid}{\texttt{jobid}\xspace}
\newcommand{\jobids}{\texttt{jobids}\xspace}
\newcommand{\datasets}{\texttt{datasets}\xspace}
\newcommand{\options}{\texttt{options}\xspace}
\newcommand{\prepare}{\texttt{prepare}\xspace}
\newcommand{\analysis}{\texttt{analysis}\xspace}
\newcommand{\synthesis}{\texttt{synthesis}\xspace}
\newcommand{\analysisres}{\texttt{analysis\_res}\xspace}
\newcommand{\prepareres}{\texttt{prepare\_res}\xspace}
\newcommand{\params}{\texttt{params}\xspace}
\newcommand{\jobparams}{\texttt{job\_params}\xspace}
\newcommand{\sliceno}{\texttt{sliceno}\xspace}
% \newcommand{\jobids}{\texttt{jobids}\xspace}

\usepackage{xspace}
%%%  A pretty minted-environment for Python  %%%
\usepackage{minted}
\usemintedstyle{colorful}
\definecolor{bg}{rgb}{0.95,0.95,0.95}
\definecolor{bg_shell}{rgb}{0.95,0.95,1.00}
\newminted[python]{python}{bgcolor=bg, frame=lines}
\newminted[pythonBEG]{python}{bgcolor=bg, frame=topline}
\newminted[pythonMID]{python}{bgcolor=bg}
\newminted[pythonEND]{python}{bgcolor=bg, frame=bottomline}
\newminted[shell]{bash}{bgcolor=bg_shell, frame=lines}

\newcommand{\exempel}[1]{\noindent\textbf{Example: #1}\\}






%\includeonly{urd}

%%%%%%%%%%%%%%%%%%%%%%%%%%%%%%%%%%%%%%%%%%%%%%%%%%%%%%%%%%%%%%%%%%%%%%%%%%%%
%                                                                          %
% Copyright (c) 2018 eBay Inc.                                             %
%                                                                          %
% Licensed under the Apache License, Version 2.0 (the "License");          %
% you may not use this file except in compliance with the License.         %
% You may obtain a copy of the License at                                  %
%                                                                          %
%  http://www.apache.org/licenses/LICENSE-2.0                              %
%                                                                          %
% Unless required by applicable law or agreed to in writing, software      %
% distributed under the License is distributed on an "AS IS" BASIS,        %
% WITHOUT WARRANTIES OR CONDITIONS OF ANY KIND, either express or implied. %
% See the License for the specific language governing permissions and      %
% limitations under the License.                                           %
%                                                                          %
%%%%%%%%%%%%%%%%%%%%%%%%%%%%%%%%%%%%%%%%%%%%%%%%%%%%%%%%%%%%%%%%%%%%%%%%%%%%

%http://tex.stackexchange.com/questions/40738/how-to-properly-make-a-latex-project
%Sadly, \input breaks the build. If pdflatex encounters a missing
%\input file, it generates an error (instead of a warning like with
%\include), and stops compiling. Yes, latexmk will generate the file
%and re-start pdflatex, but this is inefficient, and breaks completely
%if you have multiple such file references, because eventually the
%compile will end with a ``too many re-runs'' message.  John Collins'
%answer to my question regarding this provides a workaround for this.
\newcommand\inputfile[1]{%
    \InputIfFileExists{#1}{}{\typeout{No file #1.}}%
}
\newlength{\drop}% for my convenience
\newcommand*{\titleLL}{\begingroup% Lost Languages
  \drop=0.1\textheight
  \fboxsep 0.5\baselineskip
  \sffamily
  \vspace*{\drop}
  \centering
  %{\textcolor{SkyBlue}{\HUGE CONUNDRUMS}}\par
  {
    \Huge Expertmaker Accelerator
%    {\HUGE CONUNDRUMS}
  }\par
  \vspace{0.5\drop}
  %\colorbox{Dark}{\textcolor{white}{\normalfont\itshape\Large
  {
    {
      \normalfont\itshape\Large
      User's Reference
    }
  }\par
  \vspace{\drop}
  {
    \Large DRAFT\\
    \tiny Version: \inputfile{gitrevision}
  }\par
  \vfill
  {
    \footnotesize Anders Berkeman, Carl Drougge, and Sofia H\"orberg
  }\par
  \vspace*{\drop}
  \endgroup
}


\begin{document}
\titleLL
\newpage


%\tableofcontents


\section*{Glossary}

% \begin{tabular*}{\textwidth}{l @{\extracolsep{\fill}} l}
\begin{tabular*}{\textwidth}{ll}
method    & main source file of program to be executed by framework\\
package   & a location where methods are stored\\
job       & a running or completed method\\
jobid     & a link to a successfully completed method\\
workdir   & a location where job data and metadata is stored\\
dataset   & efficient storage of data\\
chain     & a one-directional list of datasets\\
urd       & dispatcher and job database\\
automata  & a script run by urd dispatching or recalling jobs\\
\end{tabular*}

--------------------------------------------------------------------------------
Wikitext:

The Accelerator is a platform capable of working at high speed with
billions of lines of data on a single computer.  Data is represented
and accessed with close to zero overhead, and results are
automatically reused.

It has been used in live projects for production and analysis since
2012.

--------------------------------------------------------------------------------

\chapter{Introduction}

The Accelerator is a very efficient environment for running and
scheduling big data tasks.  It is begin developed continuously since
the start in 2012, and has been used in seven different projects, of
which three has been live to customers, and the other four was data
analysis and insight projects.

Due to its minimalistic implementation, the Accelerator is capable of
working at high speed with billions of lines of data on a single
computer.  Two things stand out: first, all jobs are bookkeeped in a
novel way; and second, data is represented and accessed with close to
zero overhead.  The end result is a globally optimised data processing
machine with a wide spectrum of uses.

Some Accelerator features
\begin{itemize}
\item \textbf{Data integration} - The Accelerator has been used in
  several projects with different customers and data, and has thereby
  adopted to a number of formats and cases.
\item \textbf{Efficient Data Access} - Data is streamed through jobs
  using low level operating- and filesystem primitives.
\item \textbf{Simple job tracking} - Easy to find source data, easy to
  use results from other users.
\item \textbf{Transparency} - Total context for all jobs ever run is
  stored and straightforward to retreive.  Any job could be replayed
\end{itemize}





\newpage
Here is a list of some of the background ideas that coloured the work
on the Accelerator.
\begin{itemize}

\item{parallel processing}
  
\item \textbf{Data, recipies (code), and computation} - Data and code
  is more important than computation.  If data integrity is
  maintained, and code version controlled, computation is repeatable
  and can be replicated.

\item \textbf{Never delete or change input data} - What is saved by
  freed storage is nothing compared to the cost of loss of history:
  replication becomes impossible, as well as certain analyses.

\item \textbf{Data should be accessable with minimum overhead}

\item \textbf{Appending columns and rows should have minimal overhead.}

\item \textbf{Live and dev environment is equivalent}
\item \textbf{ blah - analysis and production same blah@@@}
  
\item \textbf{Bookkeep intermediate results.  Use it --- don't recompute!}

\item \textbf{client server configuration}

\item \textbf{simplicity} use basic file systems, and simple,
  minimalistic programs.  (Maintainability, stability, observability,
  and speed).  Adding layers and applications slows things down.

\item \textbf{Moore's law and bottlenecks} CPU, memory size, and disk
  storage increase exponentionally over time for constant price.
  Interconnection speed does not.  Avoid bottlenecks!

\item \textbf{Avoid clusters} Are clusters for speed or redundancy?
  How many machines does a cluster need to have to par a single
  machine with data and code stored locally?
\end{itemize}




\chapter{Overview}

\section{overview}

methods, jobs, jobids, datasets


\section{dispatch, dependencies, and jobids}
The framework is basically a job dispatcher and depencency tracker.
Input to a job is parameters and source code, called method.  When a
job is run, a directory is created in the workspace.  A pointer to
this directory, called a jobid, is returned upon finished execution.
Inside the directory, everything needed to run the job is stored, for
example input parameters, as well as all results created during
execution.

The framework keeps information for each job that has been
successfully completed for dependency tracking.  Dependency tracking
is used for two things: first, it avoids re-execution of jobs that
have been run before; and second, it permits jobs to depend on already
finished jobs.

Re-execution is avoided if a job is issued with exactly the same
source code and input options.  If source code is not equal, or if any
input to the job is different, the job will be executed again.  All
successful runs of a job will be kept.

A job may have reference to other jobs as input.  If these references
are valid, the running job will have access to all information from
the reference jobs.

\section{datasets and chaining}
A particularly efficient way to store data is by using the dataset.
Data is stored so that it can be retrieved row by row at a very high
pace and requesting from one to any number of columns.

Typically, a dataset is created for each file that is imported.  When
several files are imported, the jobids may be chained, that is, each
new jobid contains a pointer to the previous job.  Using the
previous-pointers, datasets could be iterated one after another
without interruption.




\section{methods}
A method is a piece of source code.  It may contain the following
functions:  prepare, analysis, and synthesis.

Example

\begin{python}
def prepare():
  x = 3
  return x

def analysis(sliceno, prepare_res):
  return x, sliceno
  
def synthesis(analysis_res):
  ix = 0
  for res in analysis_res:
    print res
    ix += 1
  return ix
\end{python}

\begin{python}
def synthesis(params):
  for key, val in params.items():
    print key, val

# first, the package, name of source code, jobid to this job
#   package   dev
#   method    refman_test1
# so, source code is in dev/a_refman_test1.py

# this jobid, caption and starttime
#   jobid     test-425_0
#   caption   fsm_refman_test1
#   starttime 1476105147.99
# starttime is in unix epoch time (seconds since jan 1, 1970)
    
# number of slices in current workspace
#   slices    29

# links and options
#   datasets  {'source': 'neu4-1820_0'}
#   jobids    {}
#   options   {}

# source code hash, random seed to use    
#   hash      32bbbf165d8ffdf09626a005e52c813c5ac791cb
#   seed      3399658600771568772
#   link      {}
# The seed is a convenienve providing a known random seed

# params.params containt datasets, jobids, and options
#   params    {'refman_test1': {
#                'datasets': {'source': 'neu4-1820_0'},
#                'options': {},
#                'jobids': {}}
#             }
\end{python}

a method is a program
naming convention

prepare-analysis-synthesis

params, slices


\chapter{High Level Control:  Urd}
%%%%%%%%%%%%%%%%%%%%%%%%%%%%%%%%%%%%%%%%%%%%%%%%%%%%%%%%%%%%%%%%%%%%%%%%%%%%
%                                                                          %
% Copyright (c) 2018 eBay Inc.                                             %
%                                                                          %
% Licensed under the Apache License, Version 2.0 (the "License");          %
% you may not use this file except in compliance with the License.         %
% You may obtain a copy of the License at                                  %
%                                                                          %
%  http://www.apache.org/licenses/LICENSE-2.0                              %
%                                                                          %
% Unless required by applicable law or agreed to in writing, software      %
% distributed under the License is distributed on an "AS IS" BASIS,        %
% WITHOUT WARRANTIES OR CONDITIONS OF ANY KIND, either express or implied. %
% See the License for the specific language governing permissions and      %
% limitations under the License.                                           %
%                                                                          %
%%%%%%%%%%%%%%%%%%%%%%%%%%%%%%%%%%%%%%%%%%%%%%%%%%%%%%%%%%%%%%%%%%%%%%%%%%%%

\label{chap:urd}

\section{Introduction to Urd}

Urd is the processing flow controller in the framework.  It is the
primary job dispatcher as well as the bookkeeper of all jobs executed.
Events in urd are quantified into what is called \textsl{sessions}.
The core of Urd is a transaction log database storing these sessions
together with meta information.  The result is a server providing
lookups for all jobs executed together with their context.

The Urd database is partitioned into what is called \textsl{lists}.
Lists are where information about executed jobs are stored, and how
they relate to eachother.  Lists are globally readable, but writing
requires authentication, so that, for example, only the production
user may publish a model to go live.

With the exception of experimental work, all work initiated by Urd is
run in closed sessions, with well defined starting and ending points.
The input dependencies to these sessions are recorded, together with
the resulting output.



\section{Urd Sessions}

A minimal example of creating an Urd session is as follows
\begin{python}
def main(urd):
    urd.begin('test')
    ...
    urd.finish('test', '2016-10-25')
\end{python}
Every job that is dispatched between \texttt{begin} and
\texttt{finish} will be appended to the \texttt{test} Urd list with
timestamp \texttt{20161025}.  In addition, all lookups of jobs done by
Urd will be appended to the Urd list as dependencies.

The Urd list will be updated only when the \texttt{finish} function is
called, since it is responsible for updating the urd transaction log.
Before \texttt{finish}, nothing is stored, and it is perfectly okay to
omit \texttt{finish} during development work.

There are a number of options associated with a session, as shown
here,
\begin{python}
urd.begin(path, timestamp, caption=None, update=False)
urd.finish(path, timestamp, caption=None)
\end{python}
and the following applies
\begin{itemize}
\item [] \texttt{path} is the name of the Urd list, and the same
  \texttt{path} must be specified in both \texttt{begin} and
  \texttt{finish}.

\item [] \texttt{timestamp} is mandatory, but could be set in either
  \texttt{begin}, \texttt{finish}, or both.  \texttt{finish}
  overrides \texttt{begin}.

\item [] \texttt{caption} is optional, and can be set in either
  \texttt{begin} or \texttt{finish}.  \texttt{finish} overrides
  \texttt{begin}.
\end{itemize}
There is also an \texttt{update} option that will be discussed in
section~\ref{sec:trunc-update}.



\subsection{Timestamp resolution}

Timestamps may be specified in various resolution depending on the
application.  The time format is
\begin{verbatim}
"%Y-%m-%dT%H:%M:%S"
\end{verbatim}
and it can be truncated as shown in the following examples covering all possible cases.
\begin{python}
'2016-10-25'               day resolution
'2016-10-25T15'            hour resolution
'2016-10-25T15:25'         minute resolution
'2016-10-25T15:25:00'      second resolution
\end{python}



\subsection{Aborting an Urd Session}

When an Urd session is initiated, a new session cannot be started
until the current session has finished.  A session may be aborted,
however, like this
\begin{python}
urd.begin('test')
urd.abort()
\end{python}
Aborted sessions are not stored in the Urd transaction log.




\section{Building Jobs}

Jobs are dispatched in Urd sessions using the \texttt{build} function.
The syntax is just the same as in the previous section that did not
use sessions.
\begin{python}
jobid = urd.build('method1', options={}, datasets={}, jobids={}, ...)
\end{python}
Here, \texttt{options}, \texttt{datasets}, and \texttt{jobids} are
optional, depending on the method to be dispatched.  In addition, a
name and a caption may be specified too
\begin{python}
jobid = urd.build('method1', name='myjob', caption='looking for something')
\end{python}
The name will override the default name, which is the name of the
method, in the Urd list.  In this case, this job will now be referred
to as \texttt{myjob} (instead of default \texttt{method1}).  This is
useful to separate jobs if the same method is used multiple times in
the same list.  A jobid to the finished job is returned upon
successful completion.



\subsection{Handling Consecutive Jobs}

Using the output jobid from the \texttt{build} function, it is
straightforward to connect jobs in series.  For example
\begin{python}
jid_filter = urd.build('filter_data', datasets=dict(source=<some input>))
jid_reduce = urd.build('reduce', datasets=dict(source=jid_filter))
\end{python}
In the example above, the first method, \texttt{filter\_data}, creates
a new dataset from its input.  This is then forwarded to the second
method, \texttt{reduce}, using the jobid reference
\texttt{jid\_filter}.

If the first method or its input data is changed, the first job will
be run again.  This will cause the jobid \texttt{jid\_filter} to
change too, which will, in turn, trig execution of the \texttt{reduce}
job.  The Urd list will contain jobids to both jobs, and all input
parameters as well, so it is clear which \texttt{filter\_data} job
that was used as input for the \texttt{reduce} job.



\subsection{Building Chained Jobs}

Using the \texttt{build\_chained} function, it is possible to build
chained jobs implicitly, like this
\begin{python}
jobid = urd.build_chained('method1', name='myjob')
\end{python}
This function takes the same options as the standard \texttt{build}
method, with the exception that name is mandatory, since it is used to
find the previous job of matching type.






\section{Urd Sessions with Dependencies}


\comment{forst en import-automata har, och sedan anvand den...}

A job may have dependencies, such as other jobs or datasets.  These
dependencies are input to the job using the corresponding arguments to
the \texttt{build} function.  Locating these jobs or datasets,
however, is exactly a design goal of Urd.  Urd implements a
\texttt{get} function that looks up jobids and dependencies from a key
that is composed of an Urd list name plus a timestamp.  There are
also, for convenience, \texttt{first} and \texttt{latest} functions to
get the first and latest job in an Urd list.

Here is an example.  Assume that we have a build script that imports
data files.  Information about the import jobs is stored in the
\texttt{import} Urd list.
\begin{python}
def main(urd):
    now = '20180403'
    urd.begin('import', now)
    jid_prev = urd.latest('import')
    urd.build('csvimport',
        options=dict(filename='log' + now + '.txt',),
        datasets=dict(previous=jid_prev)
    )
    urd.finish('import')
\end{python}
Note how the \texttt{previous} dataset is assigned from the output of
the \texttt{urd.latest} call, making the import jobs \textsl{chained}.
The call to \texttt{latest} will be recorded in the \texttt{import}
Urd list as well.

Now, assume that a method \texttt{computesomething}, uses these
imported datasets.  When dispatching \texttt{computesomething}, it
should be using the latest available \texttt{import}.  This example
shows again how the function \texttt{latest} is used for this purpose
\begin{python}
def main(urd):
    urd.begin('test')
    latest_import = urd.latest('import').joblist.jobid
    urd.build('computesomething', datasets=dict(source=latest_import))
    urd.finish('test', '20180403')
\end{python}
Two things have happened here.  First, urd has provided a jobid link
to the latest available \texttt{import} dataset.  Second, the
dependency of exactly this version of \texttt{import} to
\texttt{computesomething} is recorded in the urd list \texttt{test}
for timestamp \texttt{20180403}.  So, if there is a question in the
future which version of the \texttt{import} database that was used on
that date for the \texttt{computesomething} function, it is
immediately available from urd.

The more general form is \texttt{get}, which is shown below together
with its derived convenience-functions
\begin{python}
  urd.get('test', '20161001')
  urd.latest('test')
  urd.first('test')
\end{python}
And here is an example of running \texttt{computesomething} on \texttt{import} data
from previous month
\begin{python}
def main(urd):
  urd.begin('test')
  import = urd.get('import', '20160925').joblist.jobid
  urd.build('computesomething', datasets=dict(import=import))
  urd.finish('test', '20161025')
\end{python}



\section{Avoiding Recording Dependency}
Dependency-recording will be activated on use of the \texttt{get},
\texttt{latest}, and \texttt{first} functions.  If, for some reason,
the point is to just have a look at the database to see what is in
there, it can be done using the peek functions, \texttt{peek} and
\texttt{peek\_latest}, like this:
\begin{python}
urd.peek('test', '20161025')
urd.peek_latest('test')
urd.peek_first('test')
\end{python}
Note that this is in general not recommended.  These functions will
look up Urd lists with jobids that may be used to build new jobs, but
these dependencies will not be stored in the current Urd session,
causing a loss of continuity and visibility.



\section{More on Finding Items in Urd}
There is a \texttt{list} function that returns what lists are recorded
in the database:
\begin{python}
  print(urd.list())
  # ['ab/test', 'ab/live']
\end{python}
And there is also a \texttt{since} function that returns a list of all
timestamps after the input argument
\begin{python}
  print(urd.since('20161005'))
  # ["20161006", "20161007", "20161008", "20161009"]
\end{python}
The \texttt{since} is rather relaxed with respect to the resolution of
the input.  The input timestamp may be truncated from the right down
to only one digits.  An input of zero is also valid.
\begin{python}
  print(urd.since('0'))
  # ["20160101", "20161004", "20161005", "20161006", "20161007", "20161008"]
  print(urd.since('2016'))
  # ["20160101", "20161004", "20161005", "20161006", "20161007", "20161008"]
  print(urd.since('20161'))
  # ["20161004", "20161005", "20161006", "20161007", "20161008"]

  print(urd.since('2016105'))
  # ["20161006", "20161007", "20161008"]
  ...
  print(urd.since('2016105 000000'))
  # ["20161006", "20161007", "20161008"]
\end{python}



\section{Truncating and Updating}
\label{sec:trunc-update}
Since the Urd database is designed using log files, it will always
keep a consistent history of all events taken place.  It is not
possible to erase or modify old entries, but it is okay to update the
latest or set a marker in the log indicating that the list is starting
over from a certain date and everything before this marker should not
be considered anymore.

To update a list, use the \texttt{update} argument
\begin{python}
urd.begin('test', '20161025', update=True)
\end{python}
If update is True, the entry in the test list at '20161025' will be
updated, unless there has been no change.  and in order to set a
marker in the database indicating that everything before a certain
date in time should be discarded, do like this
\begin{python}
urd.truncate('test', '20160930')
\end{python}
This will rollback everything that has happened in the \texttt{test}
list back to '20160930'.  Remember, internally Urd stores the complete
history in a log file in plain text.


\section{More on Joblist and Jobtuple}

Urd is using the type joblist to keep track of successfully executed
jobs.  Each item in the joblist is of type jobtuple.  This section
will start by describing jobtuple first and then joblist.

\subsubsection{Jobtuple}

The \jobtuple type is used to group method names and corresponding
jobids.  It is basically a tuple with some extra properties, such as a
conversion of a jobtuple to \texttt{str}, which happens for example
when printing it, returns the jobid as a string.

\begin{python}
>>> jt = JobTuple('imprt', 'jid-0')

>>> jt
('imprt', 'jid-0')
\end{python}
as expected, and

\begin{python}
>>> jt.method
'imprt'

>>> jt.jobid
'jid-0'
\end{python}
but note that

\begin{python}
>>> print(jt)  # str and encode return jobid only
jid-0
\end{python}



\subsubsection{JobList}

\label{sec:joblist}
The \joblist is a list with add-ons for bookkeeping and finding jobs.
It stores instances of \jobtuple.  Here is an example.  First, define
a \jobtuple

\begin{python}
>>> jt = JobTuple('imprt', 'jid-0')
\end{python}
then define a joblist initiated with the same tuple.  Then append some
more jobs directly using the \texttt{append} method.

\begin{python}
>>> jl = JobList(jt)
>>> jl.append('learn', 'lrn-0')
>>> jl.append('imprt', 'imp-1')
\end{python}
Let's see how and what is stored in the \joblist.  The \texttt{pretty}
method is quite useful, but note that just printing the object will
show the last jobid only.

\begin{python}
>>> print(jl.pretty)
JobList(
   [  0]  imprt : jid-0
   [  1]  learn : lrn-0
   [  2]  imprt : imp-1
)

>>>  print(jl)  # jobid of latest appended JobTuple
imp-1
\end{python}

It is easy to retrieve the last job with a particular \texttt{method}
name, either by lookup or by using \texttt{find}.

\begin{python}
>>> jl['imprt']         # latest jobid with name 'imprt'
('imprt', 'imp-1')

>>> print(jl['imprt'])   # jl['imprt'] is JobTuple
imp-1
\end{python}
The Find method returns a \joblist.  Slicing also returns {\joblist}s

\begin{python}
>>> jl.find('imprt')
JobList([('imprt', 'jid-0'), ('imprt', 'imp-1')])

>>> jl[:2]
JobList([('imprt', 'jid-0'), ('learn', 'lrn-0')])
\end{python}
Looking up by index returns \jobtuple.

\begin{python}
>>> jl[0]
('imprt', 'jid-0')
\end{python}
These conveniences are also supported

\begin{python}
>>> jl.all              # list of all jobids
'jid-0,lrn-0,imp-1'

>>> jl.method           # last method
'import'

>>> jl.jobid            # last jobid
'imp-1'
\end{python}



\newpage
\section{Talking directly to Urd:  The Urd HTTP-API}

In some situations it is convenient to make calls to urd directly
without using the framework.  Urd will react to HTTP requests, so a
tool like \texttt{curl} suffice.

\noindent Show all stored lists like this
\begin{shell}
% curl http://localhost:8833/list
["ab/test"]
\end{shell}

Looking up the latest stored job in the test list
\begin{shell}
% curl http://localhost:8833/ab/test/latest
{"caption": "", "automata": "test", "user": "ab", "deps": {},
  "timestamp": "20161025", "joblist": [["method1", "test-56"],
  ["method2", "test-59"], ["method3", "test-60"]]}
\end{shell}
And see the first stored job in the test list
\begin{shell}
% curl http://localhost:8833/ab/test/first
{"caption": "", "automata": "test", "user": "ab", "deps": {},
  "timestamp": "20161025", "joblist": [["method1", "test-56"],
  ["method2", "test-59"], ["method3", "test-60"]]}
\end{shell}
See what is inside the test list stored at \texttt{20161025}
\begin{shell}
% curl http://localhost:8833/ab/test/20161025
{"caption": "", "automata": "test", "user": "ab", "deps": {},
  "timestamp": "20161025", "joblist": [["method1", "test-56"],
  ["method2", "test-59"], ["method3", "test-60"]]}
\end{shell}
And what is avaible in the test list that is more recent than \texttt{20161024}
\begin{shell}
% curl http://localhost:8833/ab/test/since/20161024
["20161025"]
\end{shell}
\begin{shell}
% curl http://localhost:8833/ab/test/since/20161026
[]
\end{shell}


\chapter{Method}
\section{Method inputs}

There are three kinds of input to a method
\begin{itemize}
\item options
\item datasets
\item jobids
\end{itemize}


\subsection*{options}
The options is of type dict.  Example

\begin{python}
from extras import OptionEnum, JobWithFile
options = dict(
  length = 1,
  colname = "gtin",
  operation = OptionEnum("lt le gt ge".ge),
  datafile = JobWithFile,
)
\end{python}
In the example above, the following happens
\begin{itemize}
\item length is typed to int, and default value is 1.
\item colname is typed to string, and default value is "gtin".
\item operation is one of lt, le, gt, ge, and defaults to ge.
\item datafile is more complex, for example datafile.jobid is a jobid,
  such as "pelle-123\_0", datafile.filename is a filename, such as
  "result.pickle", datafile.sliced is a boolean.  If true, the
  datafile is actually a sliced set of files, one for each target
  slice during execution, extra is what?
\end{itemize}

\subsubsection{types}
set, jobwithfile, datetime, date, time, timedelta




\subsubsection{JobWithFile}
\begin{python}

jobid, sliced, filename, extra


\subsubsection{OptionDefault}
What?


\subsubsection{OptionString}
Marker value to specify in options\{\} for requiring a non-empty string.
You can use plain OptionString, or you can use
OptionString('example'), without making 'example' the default.
    

\subsubsection{OptionEnum}
From the source code

\begin{python}
    """A little like Enum in python34, but string-like.                                                                                                                                                                           
    (For JSONable method option enums.)                                                                                                                                                                                           
                                                                                                                                                                                                                                  
    >>> foo = OptionEnum('a b c*')                                                                                                                                                                                                
    >>> foo.a                                                                                                                                                                                                                     
    'a'                                                                                                                                                                                                                           
    >>> foo.a == 'a'                                                                                                                                                                                                              
    True                                                                                                                                                                                                                          
    >>> foo.a == foo['a']                                                                                                                                                                                                         
    True                                                                                                                                                                                                                          
    >>> isinstance(foo.a, OptionEnumValue)                                                                                                                                                                                        
    True                                                                                                                                                                                                                          
    >>> isinstance(foo['a'], OptionEnumValue)                                                                                                                                                                                     
    True                                                                                                                                                                                                                          
    >>> foo['cde'] == 'cde'                                                                                                                                                                                                       
    True                                                                                                                                                                                                                          
    >>> foo['abc']                                                                                                                                                                                                                
    Traceback (most recent call last):                                                                                                                                                                                            
    ...                                                                                                                                                                                                                           
    KeyError: 'abc'                                                                                                                                                                                                               
                                                                                                                                                                                                                                  
    Pass either foo (for a default of None) or one of the members                                                                                                                                                                 
    as the value in options{}. You get a string back, which compares                                                                                                                                                              
    equal to the member of the same name.                                                                                                                                                                                         
                                                                                                                                                                                                                                  
    Set none_ok if you accept None as the value.                                                                                                                                                                                  
                                                                                                                                                                                                                                  
    If a value ends in * that matches all endings. You can only access                                                                                                                                                            
    these as foo['cde'] (for use in options{}).                                                                                                                                                                                   
    """
\end{python}

\section{Code flow:  prepare - analysis - synthesis}

\section{More on intermediate and result files}

\section{subjobs}


\chapter{Dataset}
\section{Introduction}
The dataset is the prefered way to store large amounts of data.  The
dataset is the container for fast and simple access to data.  Data in
a dataset are stored as a matrix in rows and columns.  Using the
dataset, data is simple to access and at a very high performance.

The dataset is implemented as a special type of job, and any method
could create a dataset.  The most obvious use of a dataset is the
cvsimport method that creates a dataset from an input file.

A job may contain any number of datasets.  This is convenient in for
example filtering jobs, where data could be split into two or more
datasets depending on the filtering function.

The dataset class is also a string, which is either ``jobid'' if only
one dataset (and name is default), or ``jobid/name'' when more than
one dataset is stored.







\section{Dataset properties}


\begin{python}
datasets = ('source',)

def synthesis():
\end{python}

\begin{python}
  print datasets.source.columns.keys()
  # [u'GTIN', u'date', u'locale', u'subsource']

  # each key, i.e. column, has a number of properties, of which the
  # most important ones are shown below
  print datasets.source.columns['locale'].type
  # ascii
  print datasets.source.columns['locale'].name
  # locale
  print datasets.source.columns['locale'].min
  # 3
  print datasets.source.columns['locale'].max
  # 9
  
\end{python}

Will procuce a vector of the number of lines in each slice, like this

\begin{python}
  print datasets.source.lines
  """
[5771L, 6939L, 6212L, 6312L, 6702L, 6341L, 5988L, 6195L,
 6741L, 6587L, 6518L, 5840L, 6327L, 5933L, 6745L, 6673L,
 6536L, 6405L, 6259L, 6455L, 6036L, 6088L, 6937L, 6245L,
 6418L, 6437L, 6360L, 6106L, 6878L]
"""
\end{python}

A tuple of number of columns and total number of lines

\begin{python}
  print datasets.source.shape
  # (4, 184984L)
\end{python}
the second number is exactly the sum of the number of lines for each
slice from above.

other properties are

\begin{python}
  print datasets.source.filename
  """
/data/incoming/raw_repository_5391.gz
"""
\end{python}

\begin{python}
  print datasets.source.caption
  """
flattening
"""
\end{python}


and more

\begin{python}
  print datasets.source.hashlabel
  """
GTIN
"""
\end{python}
\begin{python}
datasets.source.column\_filename
\end{python}

\begin{python}
  print datasets.source.previous
  """
neu4-4893_0/default
"""
\end{python}




\section{Dataset Iterators}

\subsection{iterate}
\begin{python}
# def iterate(self, sliceno, columns=None, hashlabel=None, filters=None, translators=None):
  for item in datasets.source.iterate(sliceno=None, columns=None):
    print item
\end{python}
sliceno=None iterates over all slices.
columns=None iterates over all columns.

\subsection{iterate\_chain}
\begin{python}
# def iterate_chain(self, sliceno, columns=None, length=-1, reverse=False, hashlabel=None, stop_jobid=None, pre_callback=None, post_callback=None, filters=None, translators=None):
\end{python}



\subsection{iterate\_list}
\begin{python}
  # def iterate_list(sliceno, columns, jobids, hashlabel=None, pre_callback=None, post_callback=None, filters=None, translators=None):
  from dataset import Dataset
  for item in Dataset.iterate_list(None, None, datasets.source):
    print item
\end{python}

\subsection{filters and translators}

\subsection{pre and post callback}

\section{Create New Dataset}
Datasets are either created in prepare + analysis, or in just
synthesis.

\subsection{Create in prepare + analysis}
A simple example writing three columns to the default dataset

\begin{python}
from dataset import DatasetWriter
datasets = ('previous',)

def prepare():
  dw = DatasetWriter(
    hashlabel = 'X',
    previous = datasets.previous,
  )
  dw.add('X', number)
  dw.add('Y', number)
  dw.add('Z', number)
  return dw

def analysis(sliceno, prepare_res):
  dw = prepare_res
  ...
  dw.write(x, y, z)
\end{python}
Note that the order of the variables in the dw.write function call is
the same as the order of the add calls in prepare\footnote{in case
  write is called with a dict, the order is unknown, but then names
  are looked up using the dict keys.}.

DatasetWriter takes a number of optional arguments such as caption and
filename.  The argument ``name'' specifies the name of the dataset,
which is set to be ``default'' when unassigned.  Several datasets can
be created in the same method using more than one datasetwriter
instance with different ``name''s.

There is some flexibility in the way the write function may be called

\begin{python}
  dw.write_dict({column: value})
  dw.write_list([value, value, ...])
  dw.write(value, value, ...)
  # or even
  dw.writers[name].write(value)  # return True if hashed to correct slice
\end{python}


\subsection{Creating Hashed datasets}
If hashlabel is set, one can use dw.hashcheck(value) to check if value
belongs to the slice.  It is also possible to just call the writer
since it will discard anything not belonging to the correct slice.



\subsection{Create in synthesis}

There are two options if the dataset is to be created in synthesis.
One in to set the slice number first

\begin{python}
  dw.set_slice(sliceno)
\end{python}
while the other is to use one of these functions

\begin{python}
  dw.get_split_write_dict()({column: value})
  dw.get_split_write_list()([value, value, ...])
  dw.get_split_write()(value, value, ...)
\end{python}
that will assing the data to the correct slice automatically.

\subsection{Placeholder:  Creating datasets more manually}


\section{Appending to a Dataset}
\begin{python}
datasets = ("source", "previous",)

def prepare():
  dw = dataset.DatasetWriter(
    parent=datasets.source,
    previous=datasets.previous,
    caption="flattening_attempt_1"
  )
  dw.add(name, type)
  return dw

def analysis(sliceno, prepare_res):
  dw = prepare_res
  dw.writer[name].write('x')
\end{python}



% more on datasets: previous, link_to_here
  


\chapter{Iterator}
%%%%%%%%%%%%%%%%%%%%%%%%%%%%%%%%%%%%%%%%%%%%%%%%%%%%%%%%%%%%%%%%%%%%%%%%%%%%
%                                                                          %
% Copyright (c) 2018 eBay Inc.                                             %
% Modifications copyright (c) 2019-2020 Anders Berkeman                    %
%                                                                          %
% Licensed under the Apache License, Version 2.0 (the "License");          %
% you may not use this file except in compliance with the License.         %
% You may obtain a copy of the License at                                  %
%                                                                          %
%  http://www.apache.org/licenses/LICENSE-2.0                              %
%                                                                          %
% Unless required by applicable law or agreed to in writing, software      %
% distributed under the License is distributed on an "AS IS" BASIS,        %
% WITHOUT WARRANTIES OR CONDITIONS OF ANY KIND, either express or implied. %
% See the License for the specific language governing permissions and      %
% limitations under the License.                                           %
%                                                                          %
%%%%%%%%%%%%%%%%%%%%%%%%%%%%%%%%%%%%%%%%%%%%%%%%%%%%%%%%%%%%%%%%%%%%%%%%%%%%

\label{chap:iterators}

The basic idea of the Accelerator's datasets is to make it easy to
create parallel programs that can read and write large amounts of data
at a very high speed.  High speed data read access is implemented as a
set of special Python \textsl{iterators}.  Each iterator yields
one \texttt{tuple} at a time containing elements from one or more
specified data columns, one row at a time.  In case of iterating over
a single column, the output may optionally be a scalar instead
of \texttt{tuple} for cleaner code and more efficient computing.



\section{The Three Iterators}
Technically, iterators are members of the \texttt{Dataset} class.
Iterators can be parallel, in \analysis, or sequential, in \prepare
or \synthesis.  There are three iterators available:
\begin{itemize}
\item [] \texttt{iterate()}, for single dataset iteration,
\item [] \texttt{iterate\_chain()} for iterating over dataset chains, and
\item [] \texttt{iterate\_list()} for iterating over a specified list of datasets.
\end{itemize}
And each of them will be discussed later in this chapter.  For
completeness, it should also be mentioned that
the \texttt{DatasetChain} class also has an \texttt{.iterate()}
function, and it works similar to \texttt{iterate\_chain()}.

In many common use cases it is sufficient to provide only two
arguments to the iterator: \texttt{sliceno}, which is mandatory,
and \texttt{columns}.  These, and all other arguments are presented in
detail shortly. A typical use of an iterator looks like this
\begin{python}
datasets = ('source',)

def analysis(sliceno):
    for m, u in dataset.source.iterate(sliceno, ('movie', 'user',)):
        # do something with m and u here...
\end{python}
Python's constructors can be used to create objects from iterators
like in the following example, where the purpose is to compose
a \texttt{dict}.
\begin{python}
n2d = dict(dataset.source.iterate(sliceno, ('name', 'date',)))
\end{python}
or this example
\begin{python}
from collections import Counter
...
    c = Counter(dataset.source.iterate(sliceno, 'user')))
\end{python}
that will create a counter of how many times each \texttt{user} is
present in the dataset.


\newpage
\subsection{Iterator Arguments}

All three iterators share these arguments
\starttable
  \RP \texttt{sliceno} & \textsl{mandatory} & Slice number (an
  integer) to iterate over, \mintinline{python}/None/ to iterate over
  all slices sequentially, or \texttt{roundrobin} to take one value
  per slice in a round robin fashion. \\[1ex]

  \RP \texttt{columns} & \pyNone & Tuple of column labels or a single
  name if iterating over one column.  \pyNone selects all columns in
  alphabetical order.\\[1ex]

  \RP \texttt{hashlabel} & \pyNone & Name of hash column.  If the code
  relies on a dataset being hashed on a particular column, set this to
  make the iterator \texttt{assert} that this is actually the case.
  Execution will terminate if the hashlabel is incorrect.\\[1ex]

  \RP \texttt{rehash} & \pyFalse & Setting this
  to \mintinline{python}/True/ will hash partition the dataset
  on-the-fly based on the \texttt{hashlabel} column.  (Rehashing
  on-the-fly is slower, so ideally datasets should be rehashed using
  the \texttt{dataset\_rehash}
  method~\ref{sec:dataset_hash_partition}.)\\[1ex]
  
%  \RP \texttt{translators} & \pyNone & Translators transform data
%  values.  Explained in section~\ref{sec:translators}\\[1ex]
  
%  \RP \texttt{filters} & \pyNone & Filters decide which rows to
%  include.  Explained in section~\ref{sec:filters}\\[1ex]

  \RP \texttt{status\_reporting} & \pyTrue & Give status when
  pressing \texttt{C-t}.  Unless manually \texttt{zip}ing several
  iterators together, this should be set to
  default \mintinline{python}/True/.  See \texttt{dataset.py} source
  code for full information.\\
\stoptable

\noindent In addition, \texttt{iterate\_chain} takes these arguments too
\starttable
  \RP \texttt{length}&$-1$& Number of datasets in a chain to iterate
  over.  Default is $-1$, which corresponds to all datasets in a
  chain.\\[1ex]
  
  \RP \texttt{range}& \pyNone& Filter rows based on a column's value
  being within a range, see
  section~\ref{sec:iterate_sloppy_range}\\[1ex]

  \RP \texttt{sloppy\_range}& \pyFalse & Used with \texttt{range}, but
  will iterate over full datasets for those datasets i a chain that
  have values within range, see
  section~\ref{sec:iterate_sloppy_range}.\\[1ex]
  
  \RP \texttt{reverse}& \pyFalse & Iterate chain backwards.  Default
  is to iterate forward, i.e.\ from oldest to newest dataset.\\[1ex]

  \RP \texttt{stop\_ds}& \pyNone & Iterate back to this dataset.
  Actually, setting this will iterate from the dataset
  \textsl{following} \texttt{stop\_ds} to the newest dataset in the
  chain.\\[1ex]

  \RP \texttt{pre\_callback}& \pyNone & A function that will be called
  before iterating each dataset.\\[1ex]

  \RP \texttt{post\_callback}& \pyNone & A function that will be
  called after iterating each dataset.\\
\stoptable

\noindent and \texttt{iterate\_list()} takes a \texttt{datasets} parameter
\starttable
  \RP \texttt{datasets} &\pyNone& List of datasets to iterate over.\\
\stoptable



\section{Basic Iteration}
Basic use include iterating in parallel or serial over one dataset or
a chain of datasets.


\subsection{Parallel Iterator Invocation}
For parallel iteration in \analysis, the iterator needs to know the
number of the current slice.  This information can be fed to
the \analysis function in the \texttt{sliceno} variable. The following
is an example of iteration that happens independently in each slice.
\begin{python}
from collections import defaultdict
datasets = ('source',)

def analysis(sliceno):
    h = defaultdict(set)
    for user, item in datasets.source.iterate(sliceno, columns=('user', 'item',)):
        h[user].add(item)
\end{python}
The program creates dictionaries mapping \texttt{user}s to sets of
\texttt{item}s for the \texttt{source} dataset.  Assuming that
the dataset is hash partitioned (see ~\ref{sec:slicing_and_hashing}),
this operation is entirely parallel and there is no need to merge all
the results from the analysis processes afterwards, since the
different slices do not share any keys with each other.



\subsection{Sequential Iterator Invocation}
Setting the \texttt{sliceno} parameter to
\pyNone will cause the iterator to run through all slices of
the dataset, one slice at a time, like in this example
\begin{python}
def synthesis():
    h = defaultdict(set)
    for user, item in datasets.source.iterate(None, columns=('user', 'item',)):
        h[user].add(item)
\end{python}
Dataset slices will be iterated in increasing order.



\subsection{Special Case, Round Robin Iteration}
By default, the iterators stream slices of data.  This is almost
always exactly what is needed.  But sometimes, for example when the
order of rows imported by \texttt{csvimport} matters, there is a need
for a row-wise iteration order.  For maximum performance,
the \texttt{csvimport} method writes datasets in a round robin
fashion, so iterating over a csvimported dataset does not return the
lines in the same order as they were written.

By setting the first parameter of any of the iterator functions to
``\texttt{roundrobin}'', the iterator will internally fetch all slice
iterators and return one value at a time from each iterator in a round
robin fashion.  The resulting output is then in the same order as in
the file imported by \texttt{csvimport}.  In a dataset chain, round
robin will happen \textsl{per dataset}.  There is a performance
penalty associated with this functionality, but it is handy for
time-series-like data.



\subsection{Special Cases, Iterating Over All or a Singe Column}
It is possible to iterate over all columns in a dataset by specifying
an empty list of column names, like this
\begin{python}
for items in dataset.source.iterate(sliceno, None):
    print(items)  # is a tuple of all columns
\end{python}
The iterator will output a \texttt{tuple} populated with all column
values for each row.  The columns will be in sorted column name order.

If iterating over a single column, it makes little sense to keep the
output values in a one-dimensional tuple.  A scalar is cleaner and
more efficient.  Here are the two different ways to iterate over a
single column
\begin{python}
# alternative 1, use lists/tuples
for user in datasets.source.iterate(sliceno, ('USER',)):
    userset.add(user[0])  # user is a tuple

# alternative 2, specify column as string, not list
for user in datasets.source.iterate(sliceno, 'USER',):
    userset.add(user)     # user is a scalar!
\end{python}
%Both styles are supported by filters and translators introduced later
%in this chapter.



\subsection{Iterate Over Chains}
The \texttt{iterate\_chain()} iterator is used to iterate over one or
more datasets in a chain, starting at the ``oldest'' dataset.  The
following example will iterate over the last three datasets in the
chain, oldest dataset first.
\begin{python}
datasets = ('source',)

def analysis(sliceno):
    h = defaultdict(set)
    for user, item in datasets.source.iterate_chain(
                           sliceno, columns=('user', 'item',), length=3):
        h[user].add(item)
\end{python}
Using \texttt{iterate\_chain()} without explicitly specifying
\texttt{length} will default to a \texttt{length} of $-1$, which
corresponds to all datasets in the chain.

%% Here is an interesting example of a method that will iterate over all
%% chained \texttt{source} datasets that are new since the last
%% invocation of the method.
%% \begin{python}
%% datasets = ('source',)
%% jobs = ('previous',)

%% def analysis(sliceno):
%%     h = defaultdict(set)
%%     for user, item in datasets.source.iterate_chain(
%%                            sliceno, columns=('user', 'item',),
%%                            stop_ds=jobs.previous.source,):
%%         h[user].add(item)
%% \end{python}






%% \begin{figure}[t!]
%%   \begin{center}
%%     \hspace{2.5cm}\input{figures/dsprocchain.pdftex_t}
%%     \caption{Example of import and processing jobs.  Blue text and
%%     arrows relate to the \textsl{current job}, \texttt{proc-1}.}
%%     \label{fig:dsprocchain}
%%   \end{center}
%% \end{figure}



%% \subsection{An Example}
%% Consider the case where new files are added to a project in a
%% continuous fashion.  These files are imported and chained, so that
%% each new imported dataset links to the previously imported dataset,
%% and so on.  This is illustrated in the left part of
%% figure~\ref{fig:dsprocchain}.

%% Once in a while, but not necessarily at the same rate as new files are
%% added, some processing of the data is performed.  This processing
%% could for example be \textsl{updating} a machine learning model with
%% the newly added data imported since the last model update.

%% The processing job should then iterate over a part of the the dataset
%% chain only, from the first dataset after the last processing job up to
%% the most recent imported dataset.  Again, see
%% figure~\ref{fig:dsprocchain}.  The right part of the image illustrates
%% two builds of the processing method.  The first operates on the
%% dataset \texttt{imp-0}, and the last on \texttt{imp-1}
%% and \texttt{imp-2}.  The input parameters to the first processing
%% job, \texttt{proc-0}, are
%% \begin{itemize}
%% \item[] \mintinline{python}|jobs.previous = None|
%% \item[] \mintinline{python}|datasets.source = 'imp-0'|
%% \end{itemize}
%% and the input parameters for the second processing
%% job, \texttt{proc-1}, are
%% \begin{itemize}
%% \item[] \mintinline{python}|jobs.previous = 'proc-0'|
%% \item[] \mintinline{python}|datasets.source = 'imp-2'|
%% \end{itemize}

%% The processing job should iterate on the \texttt{datasets.source}
%% dataset, and set the \texttt{stop\_id=} parameter to
%% the \texttt{previous} job's \texttt{source} dataset, and the code may
%% look something like this
%% \begin{python}
%% datasets = ('source',)
%% jobs = ('previous',)

%% def analysis(sliceno):
%%     for ... in datasets.source.iterate_chain(
%%             ...
%%             stop_ds={jobs.previous: 'source',}):
%%     ...
%% \end{python}











\section{Halting Iteration}

Iteration over a dataset chain will continue until all datasets are
exhausted or a stop criteria is fulfilled.  There are several
mechanisms for stopping, and they may be combined in a single iterator
call.  If so, iteration will be over the shortest range of the
conditions.

\subsection{Halting Using \texttt{length}}
\begin{python}
for user, item in datasets.source.iterate_chain(
                       sliceno, ('user', 'item',),
                       length = options.length):
\end{python}
This will iterate for the last \texttt{options.length} number of
datasets.  Note that a length of $-1$ is default and will iterate
without bounds.


\subsection{Halting Using \texttt{stop\_ds}}
Similar to using \texttt{length}, but will stop when reaching a
certain dataset.
\begin{python}
for user, item in datasets.source.iterate_chain(
                       sliceno, ('user', 'item',),
                       stop_ds = 'foo-3'):
\end{python}
Stopping at a constant dataset has limited value.  Next section shows
how to stop iterating based on previous jobs.



\subsection{Halting Using Another Job's Input Parameters}
\begin{python}
for user, item in datasets.source.iterate_chain(
                       sliceno, ('user', 'item',),
                       stop_ds = {jobs.previous: 'source',}):
\end{python}
This will iterate until reaching the \texttt{source} dataset of
the \texttt{jobs.previous} job.



\section{Iterating Over a Data Range}
\label{sec:iterate_sloppy_range}
It is possible to iterate over rows having a specified column's value
within a certain range.  This works best on datasets that are sorted
on the specified column.
\begin{python}
for user, item in datasets.source.iterate_chain(
           sliceno, ('user', 'item',),
           range={timestamp, datetime(2016, 1, 1), datetime(2016, 3, 31),}):
\end{python}
This example will limit the iterator to exactly the range of lines
that fulfill the range condition.  It is relatively costly to filter
each line, and there is a speed advantage by instead specifying
\texttt{sloppy\_range}, which will iterate over all datasets that
contain part of the range:
\begin{python}
for user, item in datasets.source.iterate_chain(
           sliceno, ('user', 'item',),
           sloppy_range={timestamp,
                         datetime(2016, 1, 1),
                         datetime(2016, 3, 31),}):
\end{python}
Here, all datasets that \textsl{contain} any line containing values
within the range will be included in the iteration.  Still, if the
datasets are sorted, and there are many datasets, the side-effect
caused by reading too many lines will be limited.



\section{Iterating in the Reverse Direction}
By default, iterating over a chain of dataset starts at the oldest
dataset and ends at the latest dataset.  This behavior can be
reversed by specifying \mintinline{python}/reverse=True/.  But note
that row iteration is still in the forward direction within each
dataset!
\begin{python}
for user, item in datasets.source.iterate_chain(
                       sliceno, ('user', 'item',),
                       reverse=True):
\end{python}



\section{Hash Partitioned Datasets and on-the-fly Partitioning}
Hash partitioning a dataset on a particular column, see
section~\ref{sec:slicing_and_hashing}, may really simplify the
parallel programming of methods using the dataset.  However, the
parallel code will not work properly if it turns out that the input
data is in fact not hash partitioned in the expected way.  For that
reason, it is a good idea to \emph{assert} the hashlabel by entering
it into the iterator function, like this
\begin{python}
s = {user: item for user, item datasets.source.iterate_chain(
     sliceno, ('user', 'item',), hashlabel='user')}
\end{python}
so that execution will terminate if the \texttt{hashlabel} is not correct.

It is possible to hash partition the dataset on-the-fly.  This is done
by setting the \texttt{rehash} argument to the iterator to
\mintinline{python}/True/, like this
\begin{python}
for user, item in datasets.source.iterate_chain(
                       sliceno, ('user', 'item',),
                       rehash='item'):
    # only lines with items such that
    # has(item) % slices == sliceno here
\end{python}
While this works, the preferred way to rehash is to use the
\texttt{dataset\_rehash} method~\ref{sec:dataset_hash_partition},
since it will store the rehashed dataset for later use, which in most
scenarios will be more efficient.






\section{Callbacks}
\label{sec:callback}
The iterator may be assigned callback functions that are called before
starting iterating a new dataset, and/or after the current dataset is
exhausted.  Callbacks are useful for example to aggregate data by
dataset when iterating over a large dataset chain.

There are two independent callbacks for these two cases,
called \texttt{pre\_callback} and \texttt{post\_callback}.
If \texttt{sliceno} is set to \pyNone, i.e.\ iteration runs over all
slices of all datasets in one process, it is even possible to have
callback between slice changes.

The example below will print the dataset identifier for each dataset
prior to iterating over it.
\begin{python}
# pre_callback once per dataset
def prefun(dataset):
    print(dataset.name)

for user, item in datasets.source.iterate_chain(
                       sliceno, ('user', 'item',),
                       pre_callback=prefun):
    ...
\end{python}
The argument to the callback is the dataset instance corresponding to
the dataset to be iterated next.

Next is an example of an iterator running over all slices.  The
callback function is executed before each new slice is iterated.  The
callback takes two arguments in this scenario, first, the dataset
instance as per the example above, and second the number of the slice.
\begin{python}
# callback once per slice
def prefun(dataset, sliceno):
    print(dataset.name, sliceno)

for user, item in datasets.source.iterate(
                       None, ('user', 'item',),
                       pre_callback=prefun):
    ...
\end{python}
The \texttt{post\_callback} function is defined similarly.


\subsection{Skipping Datasets and Slices from Callbacks}
It is possible to skip dataset iterations by raising exceptions, as
follows.
\begin{itemize}
\item [--] To skip the next dataset do
\begin{python}
raise SkipJob
\end{python}

\item [--]  To have the iterator skip a slice, do a
\begin{python}
raise SkipSlice
\end{python}

\item [--] And to abort iterating completely
\begin{python}
raise StopIteration
\end{python}
In this case, a \texttt{post\_callback} will never be run.
\end{itemize}



%% \section{Translators}
%% \label{sec:translators}
%% Translators transform iterator output data values on-the-fly.  A
%% translator is either a callable or a \texttt{dict}.  Translators are
%% similar to \textsl{filters} (explained later), and always
%% executed \emph{before} filtering.  The idea behind translators and
%% filters is to provide a way to modify code behavior by supplying
%% functions as options to iterators.  Using translators and filters,
%% it is possible to write re-useable functions that can have different
%% behaviour depending on context.


%% \subsection*{Callable Translator, translating \texttt{tuple}s}

%%  kolla att det funkar

%% A translator function is a function from an input \texttt{tuple} (of
%% column values) to an output \texttt{tuple} of the same length.
%% Individual items may be passed through or modified, and it is possible
%% to mix different columns with each other before sending them to the
%% iterator output.  Here is an example
%% \begin{python}
%% def merger(user, item):
%%     return ("%s:%s" % (user, item), None)

%% for merge, _ in datasets.source.iterate_chain(
%%                      None, ('user', 'item',),
%%                      translator=merger):
%%     ...
%% \end{python}
%% The purpose of this translator is to convert each
%% \mintinline{python}/(user, item)/ tuple to a string ``\texttt{user:item}''.
%% This is the first output of the translator and iterator and is stored
%% in the \texttt{merge} variable.  The second output variable is not
%% used in this application, but a variable still has to be assigned, so
%% it is set arbitrarily to \pyNone.


%% \subsection{Translator \texttt{dict}, translating columns independently}

%%  kolla att det funkar

%%  ungefar har, kolla vilka sections som har star, kolla att exempel funkar.

%% One or more columns may be translated independently using a translator
%% dictionary.  Such a dictionary is specified
%% as \mintinline{python}/{name: translation}/.  A translation may be
%% either a \texttt{dict} or a callable.  Examples of both kinds are
%% shown below.  First an example illustrating the use of a
%% translation \texttt{dict}.  Here, integers are translated into more
%% comprehensible strings.
%% \begin{python}
%% mapper = {2: 'HUMANLIKE', 4: 'LABRADOR', 5: 'STARFISH',}
%% for animal in datasets.source.iterate(
%%                    None, 'NUM_LEGS',
%%                    translator={'NUM_LEGS': mapper,}):
%%     ...
%% \end{python}
%% This translator will substitute the integers 2, 4, and 5 into strings.
%% Items missing in a translation-\texttt{dict} yield \pyNone.  The next
%% example illustrates a callable inside a translator \texttt{dict}.  In
%% this example, each \texttt{user} string is output from the iterator
%% reading right-to-left.
%% \begin{python}
%% def reverse(x):
%%     return x[::-1]
%% for resu, item in dataset.source.iterate(
%%                        None, ('USER', 'ITEM',),
%%                        translator={'USER': reverse,}):
%%     ...
%% \end{python}
%% The \texttt{item} values will be passed through, but the \texttt{user}
%% strings will be reversed.



%% \section{Filters}
%% \label{sec:filters}
%% Filter are used to decide which output data rows that are allowed to
%% reach the iterator output.  Filters are run \emph{after} translators.
%% As for translators, filter are either callables or \texttt{dict}s.


%% \subsection{Callable Filter, filtering \texttt{tuple}s}
%% Callable filters receive the iterator \texttt{tuple} as input.  The
%% output of a filter must be \pyTrue for the \texttt{tuple} to be output
%% from the iterator, otherwise the iterator skips and continues to the
%% next row.  The following two examples will iterate over all animals
%% that have at least two legs and a trunk.  The first example is without
%% filter, using an \texttt{if}-statement, and the second example uses
%% iterator \texttt{filters}.
%% \begin{python}
%% # Ex. 1. Using if-statement
%% for animal, nlegs, wtrunk in datasets.source.iterate(
%%                     None,
%%                     ('animal', 'num_legs', 'has_trunk'),
%%                     filters=filter):
%%     if nlegs>=2 and wtrunk:
%%     ...

%% # Ex. 2.  Using iterator filters
%% filter = lambda line: line[1]>=2 and line[2]

%% for animal, nlegs, wtrunk in datasets.source.iterate(
%%                     None,
%%                     ('animal', 'num_legs', 'has_trunk'),
%%                     filters=filter):
%%     ...
%% \end{python}
%% Note the indexing in the \texttt{lambda} function.  Index zero
%% corresponds to the \texttt{animal} column, which is not includes in
%% the filtering expression.



%% \subsection*{Filter Dict, filtering independent columns}

%% It is possible to filter on one or more columns independently using a
%% \texttt{dict}.  If there is more than one filter, all filters must be
%% \pyTrue for a line to be output from the iterator.
%% Below are two examples of filter \texttt{dict}s.  The first example
%% will remove all rows except the ones with valid users.
%% \begin{python}
%% # keep valid users only
%% validusers={'user1', 'user2', 'user3'}
%% filters={'user': validusers.__contains__}
%% \end{python}
%% The second example will only keep rows with valid users and movie
%% items.
%% \begin{python}
%% # keep valid users with movie items
%% validusers={'user1', 'user2', 'user3'}
%% validitems={'movie': True, 'book': False}
%% filters={'user': validusers.__contains__, 'item': validitems.get}
%% \end{python}
%% The fact that book is \pyFalse is actually redundant, since
%% missing keys will never evaluate to \pyTrue and thus result in
%% discarded lines.




%% \subsection*{Filter by Column Values}

%% Filtering could also be by column value directly.  For example, assume
%% there is a column \texttt{has\_trunk} with values being Boolean
%% integers, i.e.\ \texttt{1} or \texttt{0}.  Animals with trunks may be
%% iterated using
%% \begin{python}
%% for animal, wtrunk in datasets.source.iterate_chain(None,
%%                       ('animal', 'has_trunk',),
%%                       filters={'has_trunk': None}):
%%     trunked.add(animal)
%% \end{python}
%% This may seem strange at first, but it works because the key for
%% the \texttt{has\_trunk} column exists, and the value is
%% \pyNone, which is neither a callable nor a \texttt{dict}.






\chapter{standard methods}
\section{dataset\_type?}

\subsection{typing}

\begin{python}
float32
float64
\end{python}
\\
ignore trailing garbage
\\
\begin{python}
float32i
float64i
\end{python}
\\
floatints are integers, i.e.\ int(float(value)), used for dedotting...
\\
\begin{python}
# e = exact, error on saturate which is int32 when 32 or float64 when 64.
'floatint64e'  
'floatint32e'  
\end{python}

\begin{python}
# s = saturate
'floatint64s'  
'floatint32s'  
\end{python}

\begin{python}
# exact and ignore
'floatint64ei' 
'floatint32ei' 
\end{python}

\begin{python}
# saturate and ignore
'floatint64si' 
'floatint32si' 
\end{python}

specify base, i.e.\ arg to strtol(...,x)
\\
\begin{python}
#_0 == auto (avoid)
    'int64_0'      
    'int32_0'      
# octal
    'int64_8'      
    'int32_8'      
# decimal
    'int64_10'     
    'int32_10'     
# hexadecimal
    'int64_16'     
    'int32_16'     
\end{python}

and with ignore
\\
\begin{python}
    'int64_0i'     
    'int32_0i'     
    'int64_8i'     
    'int32_8i'     
    'int64_10i'    
    'int32_10i'    
    'int64_16i'    
    'int32_16i'    
\end{python}

\begin{python}
'strbool'
# false if in ("false, "0", "f", "no", "off", "nil", "null", "")
# true  otherwise
\end{python}

\begin{python}
#true if float has bits set to one
'floatbool'
# with ignore
'floatbooli'   
\end{python}

time and date
\\
\begin{python}
# strptime-compat format
    'datetime:*'   
    'date:*'       
    'time:*'       
    'datetimei:*'  
    'datei:*'      
    'timei:*'      
\end{python}

strings and byte sequences
\\
\begin{python}
# out from csvimport, lists of bytes
'bytes'
# with strip of characters 8-13,32 from start and end
'bytesstrip'
\end{python}

\begin{python}
# python type "unicode"
'unicode:*'    
# with strip
'unicodestrip:*
# args are
#   strict
#   replace
#   ignore
\end{python}
%python-typen unicode
%strict, replace, ignore (dataset_typing:376)
%strict:   ger error
%replace:  
%ignore:   radera dåliga tecken

\begin{python}
# sequence of letters, 7-bit representation
    'ascii'        
    'asciistrip'   
# with argument
    'ascii:*'      
    'asciistrip:*' 
# replace, strict, encode
# encode är reversibel (originalsträng kan man få tillbaka)
\end{python}

\begin{python}
# number is int or float
    'number'       
# number:int is int, but "37.0" is okay
    'number:int'  se dataset_typing.py:388
\end{python}





%\chapter{habitatiii}
%\include{habitatiii}

\end{document}


